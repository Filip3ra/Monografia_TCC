\chapter{Introdu��o}

Segundo \cite{horn86robot}, todo tri�ngulo equil�tero tem os lados iguais. J�
segundo \cite{shashua97photometric}, todo quadrado tamb�m tem.

Veja que o pacote \verb|natbib| permite uma s�rie de formas diferentes para
fazer refer�ncias bibliogr�ficas. O comando padr�o, \verb|\cite|, realiza a
cita��o comum vista no par�grafo anterior. Outros comandos permitem, por
exemplo, citar somente o autor --- por exemplo, citar o trabalho de
\citeauthor{samaras99coupled} --- ou colocar automaticamente a cita��o entre
par�nteses \citep{hougen93estimation, sato99illumination2, sato99illumination1,
sato01stability}. Os comandos usados foram, respectivamente, \verb|\citeauthor|
e \verb|\citep|. Veja a documenta��o do \verb|natbib| na Internet para conhecer
outros comandos e exemplos de uso. 

Cita��es aleat�rias para fazer com que as refer�ncias bibliogr�ficas ocupem
mais de uma p�gina: \cite{bichsel92simple, dror01statistics, guisser92new}.


\section{Motiva��o}

\dummytxtb\dummytxta

\subsection{Sub-motiva��o}


\dummytxtc\dummytxtb

\subsection{Mais uma sub-se��o}

\dummytxta\dummytxtc

\subsubsection{Descendo mais um n�vel}

\dummytxtb\dummytxta
